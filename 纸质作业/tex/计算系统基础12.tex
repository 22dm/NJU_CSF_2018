\documentclass[UTF8]{ctexart}
\usepackage{listings}
\usepackage{geometry}
\usepackage{titlesec}
\usepackage{booktabs}
\usepackage{multirow}
\usepackage{forest}
\usepackage{amsmath}
\usepackage{amssymb}
\usepackage{multicol}
\lstset{numbers=left, basicstyle=\ttfamily}
\geometry{a4paper,left=2.5cm,right=2cm,top=2.5cm,bottom=2cm,headsep=0.5cm}
\linespread{1.5}

\begin{document}
\newpagestyle{main}{
	\sethead {第十二周} {《计算系统基础》作业} {nyako}
	\setfoot {} {\thepage} {\today}
	\headrule
	\footrule
}
\pagestyle{main}
\noindent\textbf{12.6}
\begin{lstlisting}[numbers = none]
    A:      .word xFFFF0010
    ;
    START:  LW    R1, A(R0)
            LW    R2, 0(R1)
            BEQZ  R2, START
            LW    R4, 0(R1)
            J     NEXT_TASK
\end{lstlisting}

\noindent\textbf{12.7}
\begin{multicols}{2}
	\indent 输入例程:
	\begin{lstlisting}[numbers = none]
    A:      .word xFFFF0010
    B:      .word xFFFF0004
    ;
    START:  LW    R1, A(R0)
            LW    R2, 0(R1)
            ANDI  R3, R2, 2
            BEQZ  R3, START
            LW    R1, B(R0)
            LW    R4, 0(R1)
            J     NEXT_TASK
\end{lstlisting}
	
	\vfill\null
	\columnbreak
	
	\indent 输出例程:
	\begin{lstlisting}[numbers = none]
    A:      .word xFFFF0010
    B:      .word xFFFF000C
    ;
    START:  LW    R1, A(R0)
            LW    R2, 0(R1)
            ANDI  R3, R2, 1
            BEQZ  R3, START
            LW    R1, B(R0)
            SW    0(R1), R4
            J     NEXT_TASK
\end{lstlisting}
\end{multicols}

\noindent\textbf{12.8} 在屏幕上显示a到z.\\

\noindent\textbf{13.3} 1). \texttt{TRAP x0C}\\
2). 可以,在进入\texttt{GETC}与\texttt{OUT}两个例程之前,\texttt{R31}已被储存,返回后,程序仍可以按照正常顺序执行.\\

\noindent\textbf{13.4} 1). 每个服务例程占用$2^8=256$个存储单元,256个共占用$2^8*256=2^{16}=65536$个存储单元.\\
2). 占用$2^{10}*256=2^{18}=262144$个存储单元.\\

\noindent\textbf{13.5} 程序开始将\texttt{x00000018}到\texttt{x0000001B}中存入了\texttt{0},造成在执行\texttt{TRAP x06}例程时,程序无法从TRAP向量表中找到该例程的起始地址,导致出错.\\

\noindent\textbf{13.7} \texttt{abcd}\\

\noindent\textbf{13.10} 1). 不断打印A,当任意键被按下时,有可能输出被按下的字符(取决于按下时,程序是否已经执行完\texttt{ADDI R4, R0, x41}),并继续输出A.\\
2). 程序会不断打印A,当8被按下时,有可能输出8或者88(取决于按下时,程序是否已经执行完\texttt{ADDI R4, R0, x41}),并继续输出A.\\

\noindent\textbf{13.13} 1).
\begin{lstlisting}[language=C, numbers = none]
    printf("(%d)-%d-%d", a, b, c);
\end{lstlisting}
\indent 2).
\begin{lstlisting}[language=C, numbers = none]
    int a, b, c;
    scanf("(%d)%d,%d", &a, &b, &c);
\end{lstlisting}

\noindent\textbf{13.14} 1). \texttt{0201\textvisiblespace}\\
2). \texttt{1020CD}\\
3). \texttt{001C}\\

\noindent\textbf{14.7} 判断\texttt{NUM}是否为素数,如果是,\texttt{RESULT}为\texttt{1},否则为\texttt{0}.\\

\noindent\textbf{14.8} 将\texttt{DATA}中的数据按照从小到大排序(冒泡排序).\\

\noindent\textbf{14.9} \texttt{R31}的值在进入\texttt{SubB}后会改变,导致无法从\texttt{SubA}中返回\texttt{main},可作如下改动:
\begin{lstlisting}[numbers = none, tabsize = 2]
	         .data   x30000000
	NUM:     .word   #8
+ SaveR31: .space  #4
	;
					 .text   x40000000
					 .global main
	main:    JAL     SubA
					 TRAP    x07
					 TRAP    x00
	;
	SubA:		 LW			 R1, NUM(R0)
+ 				 SW			 SaveR31(R0), R31
					 JAL		 SubB
+ 				 LW      R31, SaveR31(R0)
					 RET
	;
	SubB:    ADDI		 R4, R1, x30
					 RET
\end{lstlisting}

\noindent\textbf{14.10} 屏幕显示\texttt{Please enter your string:},用户输入一个字符串,每输入一个字符,屏幕同时回显该字符,直到遇到回车或长度达到100结束,随后输出这个字符串的逆序.\\

\noindent\textbf{11.18}
\begin{center}
	\begin{tikzpicture}
		\draw[draw=none] (-1,0) rectangle (5,-1);
		\draw (0,0)--(5,0);
		\draw (0,-1)--(5,-1);
		
		\foreach \i in {0,...,5}{
				\draw (\i,0)--++(0,-1);
			}
		
		\node (b) at (-0.5,-0.5) {};
		\node (b0) at (0.5,-0.5) {};
		\node (b1) at (1.5,-0.5) {};
		\node (b2) at (2.5,-0.5) {};
		\node (b3) at (3.5,-0.5) {};
		\node (b4) at (4.5,-0.5) {};
		
		\draw[->] (-0.5,-1.5)--++(0,0.5);
		\node (p) at (-0.5,-1.8) {栈顶};
	\end{tikzpicture}
\end{center}
\begin{center}
	\begin{tikzpicture}
		\draw[draw=none] (-1,0) rectangle (5,-1);
		\draw (0,0)--(5,0);
		\draw (0,-1)--(5,-1);
		
		\foreach \i in {0,...,5}{
				\draw (\i,0)--++(0,-1);
			}
		
		\node (b) at (-0.5,-0.5) {};
		\node (b0) at (0.5,-0.5) {1};
		\node (b1) at (1.5,-0.5) {};
		\node (b2) at (2.5,-0.5) {};
		\node (b3) at (3.5,-0.5) {};
		\node (b4) at (4.5,-0.5) {};
		
		\draw[->] (0.5,-1.5)--++(0,0.5);
		\node (p) at (0.5,-1.8) {栈顶};
	\end{tikzpicture}
\end{center}
\begin{center}
	\begin{tikzpicture}
		\draw[draw=none] (-1,0) rectangle (5,-1);
		\draw (0,0)--(5,0);
		\draw (0,-1)--(5,-1);
		
		\foreach \i in {0,...,5}{
				\draw (\i,0)--++(0,-1);
			}
		
		\node (b) at (-0.5,-0.5) {};
		\node (b0) at (0.5,-0.5) {1};
		\node (b1) at (1.5,-0.5) {2};
		\node (b2) at (2.5,-0.5) {};
		\node (b3) at (3.5,-0.5) {};
		\node (b4) at (4.5,-0.5) {};
		
		\draw[->] (1.5,-1.5)--++(0,0.5);
		\node (p) at (1.5,-1.8) {栈顶};
	\end{tikzpicture}
\end{center}
\begin{center}
	\begin{tikzpicture}
		\draw[draw=none] (-1,0) rectangle (5,-1);
		\draw (0,0)--(5,0);
		\draw (0,-1)--(5,-1);
		
		\foreach \i in {0,...,5}{
				\draw (\i,0)--++(0,-1);
			}
		
		\node (b) at (-0.5,-0.5) {};
		\node (b0) at (0.5,-0.5) {1};
		\node (b1) at (1.5,-0.5) {2};
		\node (b2) at (2.5,-0.5) {3};
		\node (b3) at (3.5,-0.5) {};
		\node (b4) at (4.5,-0.5) {};
		
		\draw[->] (2.5,-1.5)--++(0,0.5);
		\node (p) at (2.5,-1.8) {栈顶};
	\end{tikzpicture}
\end{center}
\begin{center}
	\begin{tikzpicture}
		\draw[draw=none] (-1,0) rectangle (5,-1);
		\draw (0,0)--(5,0);
		\draw (0,-1)--(5,-1);
		
		\foreach \i in {0,...,5}{
				\draw (\i,0)--++(0,-1);
			}
		
		\node (b) at (-0.5,-0.5) {};
		\node (b0) at (0.5,-0.5) {1};
		\node (b1) at (1.5,-0.5) {2};
		\node (b2) at (2.5,-0.5) {3};
		\node (b3) at (3.5,-0.5) {};
		\node (b4) at (4.5,-0.5) {};
		
		\draw[->] (1.5,-1.5)--++(0,0.5);
		\node (p) at (1.5,-1.8) {栈顶};
	\end{tikzpicture}
\end{center}
\begin{center}
	\begin{tikzpicture}
		\draw[draw=none] (-1,0) rectangle (5,-1);
		\draw (0,0)--(5,0);
		\draw (0,-1)--(5,-1);
		
		\foreach \i in {0,...,5}{
				\draw (\i,0)--++(0,-1);
			}
		
		\node (b) at (-0.5,-0.5) {};
		\node (b0) at (0.5,-0.5) {1};
		\node (b1) at (1.5,-0.5) {2};
		\node (b2) at (2.5,-0.5) {4};
		\node (b3) at (3.5,-0.5) {};
		\node (b4) at (4.5,-0.5) {};
		
		\draw[->] (2.5,-1.5)--++(0,0.5);
		\node (p) at (2.5,-1.8) {栈顶};
	\end{tikzpicture}
\end{center}
\begin{center}
	\begin{tikzpicture}
		\draw[draw=none] (-1,0) rectangle (5,-1);
		\draw (0,0)--(5,0);
		\draw (0,-1)--(5,-1);
		
		\foreach \i in {0,...,5}{
				\draw (\i,0)--++(0,-1);
			}
		
		\node (b) at (-0.5,-0.5) {};
		\node (b0) at (0.5,-0.5) {1};
		\node (b1) at (1.5,-0.5) {2};
		\node (b2) at (2.5,-0.5) {4};
		\node (b3) at (3.5,-0.5) {5};
		\node (b4) at (4.5,-0.5) {};
		
		\draw[->] (3.5,-1.5)--++(0,0.5);
		\node (p) at (3.5,-1.8) {栈顶};
	\end{tikzpicture}
\end{center}
\begin{center}
	\begin{tikzpicture}
		\draw[draw=none] (-1,0) rectangle (5,-1);
		\draw (0,0)--(5,0);
		\draw (0,-1)--(5,-1);
		
		\foreach \i in {0,...,5}{
				\draw (\i,0)--++(0,-1);
			}
		
		\node (b) at (-0.5,-0.5) {};
		\node (b0) at (0.5,-0.5) {1};
		\node (b1) at (1.5,-0.5) {2};
		\node (b2) at (2.5,-0.5) {4};
		\node (b3) at (3.5,-0.5) {5};
		\node (b4) at (4.5,-0.5) {};
		
		\draw[->] (2.5,-1.5)--++(0,0.5);
		\node (p) at (2.5,-1.8) {栈顶};
	\end{tikzpicture}
\end{center}
\begin{center}
	\begin{tikzpicture}
		\draw[draw=none] (-1,0) rectangle (5,-1);
		\draw (0,0)--(5,0);
		\draw (0,-1)--(5,-1);
		
		\foreach \i in {0,...,5}{
				\draw (\i,0)--++(0,-1);
			}
		
		\node (b) at (-0.5,-0.5) {};
		\node (b0) at (0.5,-0.5) {1};
		\node (b1) at (1.5,-0.5) {2};
		\node (b2) at (2.5,-0.5) {4};
		\node (b3) at (3.5,-0.5) {5};
		\node (b4) at (4.5,-0.5) {};
		
		\draw[->] (1.5,-1.5)--++(0,0.5);
		\node (p) at (1.5,-1.8) {栈顶};
	\end{tikzpicture}
\end{center}
\end{document}

