\documentclass[UTF8,fontset=fandol]{ctexart}
\usepackage{listings}
\usepackage{geometry}
\usepackage{titlesec}
\usepackage{booktabs}
\usepackage{multirow}
\usepackage{forest}
\usepackage{amsmath}
\usepackage{amssymb}
\usepackage{multicol}
\lstset{numbers=left, basicstyle=\ttfamily}
\geometry{a4paper,left=2.5cm,right=2cm,top=2cm,bottom=2cm,headsep=0.5cm}
\linespread{1.5}

\begin{document}
\newpagestyle{main}{
    \sethead {第十五周} {《计算系统基础》作业} {nyako}
    \setfoot {} {\thepage} {\today}
    \headrule
    \footrule
}
\pagestyle{main}
\noindent\textbf{17.1}\\
\indent 1). 求m与n的最大公约数.\\
\indent 2). 求$\lfloor\log_n m\rfloor$.\\
\noindent\textbf{17.2}\\
\indent 1). 计算$1+2+\cdots+x$.\\
\indent 2).
\linespread{1}
\begin{lstlisting}[language=C,numbers=none]
        int Func(int x) {
            int sum = 0;
            while (x != 0) {
                sum += x;
                x--;
            }
            return sum;
        }
\end{lstlisting}
\linespread{1.5}

为了不溢出,x最大为$2^{16}-1$.\\
\indent 3).
\linespread{1}
\begin{lstlisting}[numbers=none]
        FUNC:   SUBI  R29, R29, #4
                SW    0(R29), R4
                SUBI  R29, R29, #4
                SW    0(R29), R31
                ;
                BEQZ  R4, BASE
                SUBI  R4, R4, #1
                JAL   FUNC
                ADDI  R4, R4, #1
                ADD   R2, R2, R4
                J     END
        BASE:   ADDI  R2, R0, #0
                ;
        END:    LW    R31, 0(R29)
                ADDI  R29, R29, #4
                LW    R4, 0(R29)
                ADDI  R29, R29, #4
                RET
\end{lstlisting}
\linespread{1.5}

\indent 4). 运行时栈共$1\text{ KB}\div4\text{ B}=256$字,每次递归需要2字,故x最大为127.
\end{document}
