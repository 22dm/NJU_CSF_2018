\documentclass[UTF8]{ctexart}
\usepackage{listings}
\usepackage{geometry}
\usepackage{titlesec}
\usepackage{booktabs}
\usepackage{multirow}
\usepackage{forest}
\usepackage{amsmath}
\usepackage{amssymb}
\usepackage{multicol}
\lstset{numbers=left, basicstyle=\ttfamily}
\geometry{a4paper,left=2.5cm,right=2cm,top=2cm,bottom=2cm,headsep=0.5cm}
\linespread{1.5}

\begin{document}
\newpagestyle{main}{
    \sethead {第十三周} {《计算系统基础》作业} {nyako}
    \setfoot {} {\thepage} {\today}
    \headrule
    \footrule
}
\pagestyle{main}
\noindent\textbf{15.1}\quad\texttt{4 1 -1}\\
\noindent\textbf{15.2}\quad\texttt{24 288 27648}\\
\noindent\textbf{15.3}\\
\indent\texttt{x = 1, y = 2}\\
\indent\texttt{x = 2, y = 1}\\
\indent\texttt{x = 1, y = 2}\\
\noindent\textbf{15.9}
\linespread{1}
\begin{lstlisting}[language=C]
    #include <stdio.h>

+   int Func1(int, int);
+   int Func2(int);

    int main() {
        int x = 1;
        int y = 2;

        x = Func1(x, y);
        y = Func2(y);

        printf("x = %d y = %d\n", x, y);
    }

-   int Func1(int x) {
+   int Func1(int x, int y) {
        return x + y;
    }

    int Func2(int x) {
        int y;
?       //此处的 y 没有初始化,但不知 Func2 的具体功能,因而无法判断.
        return x - y;
    }
\end{lstlisting}

\noindent\textbf{15.7}
\begin{lstlisting}[numbers = none]
ToUpper:  SUBI  R29, R29, #4
          SW    0(R29), R30
          ADDI  R30, R29, #4
          SUBI  R29, R29, #4
          SW    0(R29), R16
          ;
          SLTI  R8, R4, x61
          BNEZ  R8, J1
          SLEI  R8, R4, x7A
          BEQZ  R8, J1
          SUBI  R4, R4, x20
J1:       ADDI  R16, R4, #0
          ADDI  R2, R16, #0
          ;
          LW    R16, 0(R29)
          ADDI  R29, R29, #4
          LW    R30, 0(R29)
          ADDI  R29, R29, #4
          RET
\end{lstlisting}

\pagebreak

\linespread{1.5}
\noindent\textbf{16.1}\\
\indent 1). \texttt{5}\\
\indent 2). 运行时栈中有三个变量x, ptr1, ptr2,其中,ptr2的值为ptr1的地址,ptr1的值为x的地址,x的值为5.\\

\noindent\textbf{16.2}\\
\indent 1). \texttt{9}\\
\indent 2). \texttt{HELLO}

\noindent\textbf{16.9}
\linespread{1}
\begin{lstlisting}[numbers = none]
StringLength:   SUBI  R29, R29, #4
                SW    0(R29), R16
                ;
                ADDI  R16, R0, #0
J1:             ADDI  R8, R4, R16
                LB    R9, 0(R8)
                BEQZ  R9, J2
                ADDI  R16, R16, #1
                J     J1
J2:             ADDI  R2, R16, #0
                ;
                LW    R16, 0(R29)
                ADDI  R29, R29, #4
                RET
\end{lstlisting}

\noindent\textbf{16.10}
\begin{lstlisting}[language=C]
    #include <stdio.h>

    char* ToUpper(char* inchar);
  
    int main() {
        char str[10];
  
        printf("Enter a string: ");
        scanf("%s", str);
  
        printf("%s\n", ToUpper(str));
    }
  
    char* ToUpper(char* inchar) {
-       char str[10];
+       char* str = inchar;
        int i = 0;
        while (*(inchar + i) != '\0') {
            if ('a' <= *(inchar + i) && *(inchar + i) <= 'z')
                *(str + i) = *(inchar + i) - ('a' - 'A');
            else
                *(str + i) = *(inchar + i);
            i++;
        }
        return str;
    }
\end{lstlisting}

\linespread{1.5}

\noindent\textbf{16.14}\quad\texttt{abc123}

\end{document}