\documentclass[UTF8]{ctexart}
\usepackage{listings} 
\usepackage{geometry}
\usepackage{titlesec}
\usepackage{booktabs}
\usepackage{multirow}
\usepackage{amsmath}
\usepackage{amssymb}
\lstset{numbers=left, basicstyle=\ttfamily}
\geometry{a4paper,left=2.5cm,right=2cm,top=2.5cm,bottom=2cm,headsep=0.5cm}
\linespread{1}

\begin{document} 

\newpagestyle{main}{            
    \sethead {第六周} {《计算系统基础》作业} {nyako}
    \setfoot {} {\thepage} {\today}
    \headrule
    \footrule
}
\pagestyle{main}
\paragraph{5.1\\}
1).
\begin{lstlisting}[language=C]
-   #define TRUE 1;
+   #define TRUE 1

    if (TRUE)
        printf("True");
    else
        printf("False");

\end{lstlisting}
2).
\begin{lstlisting}[language=C]
-   if (x == 0)
+   if (x == 0) {
        a++;
        b--;
-   else
+   }
+   else {
        a--;
        b++;
+   }
\end{lstlisting}
\paragraph{5.2\\}
1).
\begin{lstlisting}[language=C]
    #include <stdio.h>

    int main() {
        int i = 0;
        int sum = 0;

-       for (; i >= 10; ++i)
+       for (; i <= 10; ++i)
            sum = sum + i;
        printf("%d\n", sum);
    }
\end{lstlisting}
2).
\begin{lstlisting}[language=C]
    #include <stdio.h>

    int main() {
        int i = 0;
        int sum = 0;

-       while (i < 10)
+       while (i <= 10)
            sum = sum + i++;
        printf("%d\n", sum);
    }
\end{lstlisting}
3).
\begin{lstlisting}[language=C]
    #include <stdio.h>

    int main() {
        int i;
        int sum = 0;

        for (i = 0; i <= 10;)
-           sum = sum + ++i ;
+           sum = sum + i++;
        printf("%d\n", sum);
    }
\end{lstlisting}
4).
\begin{lstlisting}[language=C]
    #include <stdio.h>

    int main() {
        int i;
        int sum = 0;

        for (;;)
            if (i < 10) {
                sum = sum + ++i;
-               printf("%d\n", sum);
            }
            else
            break;
+       printf("%d\n", sum);
    }
\end{lstlisting}
\paragraph{5.3}
\begin{lstlisting}[language=C]
    #include <stdio.h>

    int main() {
        int valueA, valueB;
        int num, divisor;
        int prime;

        printf("Enter the valueA:");
        scanf("%d", &valueA);

        printf("Enter the valueB:");
        scanf("%d", &valueB);

        for (num = valueA; num <= valueB; num++) {
            prime = 1;

-           for (divisor = 2; divisor <= num; divisor++)
+           for (divisor = 2; divisor < num; divisor++)
                if ((num % divisor) == 0) {
                    prime = 0;
                }
            if (prime)
                printf("The number %d is prime\n", num);
        }
    }
\end{lstlisting}
\paragraph{5.4}
\begin{lstlisting}[language=C]
    #include <stdio.h>

    int main() {
+       int year;
        int month;
        int day;
        int sum;

-       printf("Input the date(month day):");
-       scanf("%d%d", &month, &day);
+       printf("Input the date(year month day):");
+       scanf("%d%d%d", &year, &month, &day);

        switch (month) {
            case 1: sum = day; break;
            case 2: sum = day + 31; break;
            case 3: sum = day + 59; break;
            case 4: sum = day + 90; break;
            case 5: sum = day + 120; break;
            case 6: sum = day + 151; break;
            case 7: sum = day + 181; break;
            case 8: sum = day + 212; break;
            case 9: sum = day + 243; break;
            case 10: sum = day + 273; break;
            case 11: sum = day + 304; break;
            case 12: sum = day + 334; break;
        }
+       sum += !(year % (year % 100 ? 4 : 400)) && month > 2;

-       printf("%d.%d is the %dth days.\n", month, day, sum);
+       printf("%d.%d.%d is the %dth days.\n", year, month, day, sum);
    }
\end{lstlisting}
\end{document}