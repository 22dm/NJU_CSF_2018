\documentclass[UTF8]{ctexart}
\usepackage{listings} 
\usepackage{geometry}
\usepackage{titlesec}
\usepackage{booktabs}
\usepackage{amsmath}
\usepackage{amssymb}
\lstset{numbers=left, basicstyle=\ttfamily}
\geometry{a4paper,left=2.5cm,right=2cm,top=2.5cm,bottom=2cm,headsep=0.5cm}
\linespread{1}

\begin{document}

\newpagestyle{main}{
	\sethead {第三周} {《计算系统基础》作业} {nyako}
	\setfoot {} {\thepage} {\today}
	\headrule
	\footrule
}
\pagestyle{main}

\paragraph{1.2.}
一样多,通用计算机都可解决相同的问题.

\paragraph{1.6.}
1).
令$sum=0$,对于$i$从$1$到$10$,分别用$sum$加上$i$,$sum$即为所求.
\\\\
2).
如果年份是$400$的倍数,或者是$4$的倍数但不是$100$的倍数.
\\\\
3).
如果对所有的$2\le i\le n$,均有$i\nmid n$,则$n$为素数,否则为合数.

\paragraph{3.}
\begin{center}
	\begin{tabular}{cccc}
		\toprule
		问题       & 明天下雨的概率 & 今天是星期几 & 求图形的面积 \\
		\midrule
		算法       & 将两个数相加   & 辗转相除法   & 深度优先搜索 \\
		\midrule
		程序       & Chrome         & Word         & Photoshop    \\
		\midrule
		语言处理   & GCC            & JVM          & CPython      \\
		\midrule
		操作系统   & Windows        & Linux        & BSD          \\
		\midrule
		指令集结构 & x86            & ARM          & MIPS         \\
		\midrule
		微处理器   & Intel 4004     & MOS 6502     & Intel 8086   \\
		\midrule
		逻辑电路   & 加法器         & 译码器       & 寄存器       \\
		\midrule
		元件       & 晶体管         & 电容         & 电阻         \\
		\bottomrule
	\end{tabular}
\end{center}
\end{document}