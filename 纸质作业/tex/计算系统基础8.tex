\documentclass[UTF8]{ctexart}
\usepackage{listings}
\usepackage{geometry}
\usepackage{titlesec}
\usepackage{booktabs}
\usepackage{multirow}
\usepackage{amsmath}
\usepackage{amssymb}
\usepackage{multicol}
\lstset{numbers=left, basicstyle=\ttfamily}
\geometry{a4paper,left=2.5cm,right=2cm,top=2.3cm,bottom=2cm,headsep=0.5cm}


\begin{document}

\newpagestyle{main}{
    \sethead {第八周} {《计算系统基础》作业} {nyako}
    \setfoot {} {\thepage} {\today}
    \headrule
    \footrule
}
\pagestyle{main}
\linespread{1.5}
\noindent\textbf{8.2}\\
\indent1). 单元0包含$0000\ 0000_2$,单元4包含$0100\ 0010_2$.\\
\indent2).\\
\indent\indent\textcircled{\scriptsize{1}}. 单元0:0,单元1:$-2$\\
\indent \indent\textcircled{\scriptsize{2}}. 单元2:128,单元3:127\\
\indent\indent\textcircled{\scriptsize{3}}. 单元4:B\\
\indent\indent\textcircled{\scriptsize{4}}. $\rm 0\ 1000\ 0101\ 101\ 0101\ 1000\ 0000\ 0000\ 0000_{IEEE\ 754}=106.75$\\
\indent3). 将R3和R4里的内容相加,结果存回R5里.\\
\indent4). 单元11指向单元1,其中的数为$1111\ 1110_2$.\\
\noindent\textbf{8.3} 可知补码整数有$16-\lceil\log_{2}{12}\rceil-2\times\lceil\log_{2}{8}\rceil=6$位,范围为$-2^{5}$到$2^{5}-1$,即$-32$到$31$.\\
\noindent\textbf{8.4} 可知无符号整数有$32-\lceil\log_{2}{200}\rceil-3\times\lceil\log_{2}{60}\rceil=6$位,最大数为$2^{6}-1$,即$63$.\\
\noindent\textbf{9.1}\\
\indent 1). 立即数有16位,范围为$-2^{15}$到$2^{15}-1$,即$-32768$到$32767$.\\
\indent 2). 范围为$0$到$2^{16}-1$,即$0$到$65535$.\\
\indent 3). 所能加载的最大地址为$\rm 0x4000\ 0000 + 0xFFFC +0x3= 0x4000\ FFFF$.\\
\indent 4). 表示寄存器所需的位数减1,故立即数位数增加2,可表示的最大值为$2^{17}-1$,即$133071$.\\
\noindent\textbf{9.2} R - 类型指令有5个未用位,增加寄存器需要多用$3\times(\log_{2}{128}-\log_{2}{32})=6$位,故不可以.\\
\noindent\textbf{9.3}\\
\indent 1). 需要$\log_{2}{65536}=16$位表示地址.\\
\indent 2). 偏移值为$20-(10+1)=9$.\\
\noindent\textbf{9.4} \texttt{001101 01000 01001 0000 0000 0000 0011 (SLLI R9, R8, \#3)}\\
\noindent\textbf{9.5}\\
\indent 1). \texttt{000001 00001 00010 0000 0000 0000 0000 (ADDI R2, R1, \#0)}\\
\indent 2). \texttt{001011 00001 00001 1111 1111 1111 1111 (XORI R1, R1, \#0xFFFF)}\\
\indent 3). \texttt{001001 00001 00001 0000 0000 0000 0011 (ANDI R1, R1, \#3)}\\
\noindent\textbf{9.9}
\begin{center}
    \texttt{$\rm R_1$=0x4000 0000}\quad\quad
    \texttt{$\rm R_2$=0x4321 0000}\quad\quad
    \texttt{$\rm R_3$=0x0000 0043}\\
    \texttt{$\rm R_4$=0x0000 8000}\quad\quad
    \texttt{$\rm R_5$=0x0000 0021}\quad\quad
    \texttt{$\rm R_6$=0x0000 4000}\\
\end{center}
\noindent\textbf{9.10} 前三次计算后,$\rm R_{3}=\overline{\overline{R_{1}}+R_{2}}$,跳转说明$\rm R_{3}=0$,即$\rm \overline{R_{1}}+R_{2}=-1$,所以$\rm R_{1}=R_{2}$.\\
\noindent\textbf{9.11} 此程序等价于
\linespread{1}
\begin{lstlisting}[language=C]
R1 = 0;
R3 = 16;
R4 = 1;
do {
    R5 = R2 & R4;
    if (R5 != 0)
        R1 = R1 + 1;
    R4 = R4 << 1;
    R3 = R3 - 1;
} while (R3 != 0);
    \end{lstlisting}
\linespread{1.5}
即计算$\rm R_2$的低16位中1的个数,因此可知$\rm R_2$的低16位中有7个1.
\end{document}